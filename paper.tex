\documentclass[a4paper]{article}

\usepackage{INTERSPEECH2016}

\usepackage{graphicx}
\usepackage{amssymb,amsmath,bm}
\usepackage{textcomp}
\usepackage{todo}

\def\vec#1{\ensuremath{\bm{{#1}}}}
\def\mat#1{\vec{#1}}


\sloppy % better line breaks
\ninept

\title{Multimodal Corpora}

% Authors
\makeatletter
\def\name#1{\gdef\@name{#1\\}}
\makeatother
\name{{\em Tuan Pham Minh, Nils Harder, Tilman Krokotsch}}

% Affilations and Contact information
\address{Otto-von-Guericke-University Magdeburg \\
  Faculty of Computer Science \\
  Faculty of Electrical Engineering and Information Technology \\
  {\small \tt \{tuan.pham, nils.harder, tilman.krokotsch\}@st.ovgu.de}
}

%
\begin{document}

	\maketitle
  	%
  	\begin{abstract}
    	This paper is a short summary of three multimodal corpora for the course MMK in the winter term 16/17 at the Otto-von-Guericke-University Magdeburg. The concept and aim of multimodal corpora is explained. Afterwards three exemplary corpora are described. This includes the recording process and the proposed application context, as well as the quality and quantity of the data.
  	\end{abstract}
  	\noindent{\bf Index Terms}: human machine interaction, multimodal corpora, computional linguistics

	% Introduction to the definitions and principles of multimodal data
	
  	\section{Introduction}
		\Todo{Add definitions and intro to multimodal corpora}

	\section{First Corpus}
		\Todo{Add first corpus}
	
	\section{Second Corpus}
		\Todo{Add second corpus}
	
	\section{The Bielefeld Speech and Gesture Alignment Corpus ( SaGA )}
	
		\subsection{Recording Scenario}
		
		\subsection{Recorded Data}
		
		\subsection{Evaluation}
		
		\subsection{Application}

  	\newpage
  	\eightpt
  	\bibliographystyle{IEEEtran}

  	\bibliography{thebib}

\end{document}
