\documentclass[a4paper]{article}

\usepackage{INTERSPEECH2016}

\usepackage{graphicx}
\usepackage{amssymb,amsmath,bm}
\usepackage{textcomp}

\def\vec#1{\ensuremath{\bm{{#1}}}}
\def\mat#1{\vec{#1}}


\sloppy % better line breaks
\ninept

\title{Multimodal Corpora}

%%%%%%%%%%%%%%%%%%%%%%%%%%%%%%%%%%%%%%%%%%%%%%%%%%%%%%%%%%%%%%%%%%%%%%%%%%
%% If multiple authors, uncomment and edit the lines shown below.       %%
%% Note that each line must be emphasized {\em } by itself.             %%
%% (by Stephen Martucci, author of spconf.sty).                         %%
%%%%%%%%%%%%%%%%%%%%%%%%%%%%%%%%%%%%%%%%%%%%%%%%%%%%%%%%%%%%%%%%%%%%%%%%%%
\makeatletter
\def\name#1{\gdef\@name{#1\\}}
\makeatother
\name{{\em Tuan Pham Minh, Nils Harder, Tilman Krokotsch}}
%%%%%%%%%%%%%%% End of required multiple authors changes %%%%%%%%%%%%%%%%%

\address{Otto-von-Guericke-University Magdeburg \\
  Faculty of Computer Science \\
  Faculty of Electrical Engineering and Information Technology \\
  {\small \tt \{tuan.pham, nils.harder, tilman.krokotsch\}@st.ovgu.de}
}

%
\begin{document}

  \maketitle
  %
  \begin{abstract}
    This paper is a short summary of three multimodal corpora for the course MMK in the winter term
    16/17 at the Otto-von-Guericke-University Magdeburg. The concept and aim of multimodal corpora is 
    explained. Afterwards three exemplary corpora are described. This includes the recording process 
    and the proposed application context, as well as the quality and quantity of the data.
  \end{abstract}
  \noindent{\bf Index Terms}: human machine interaction, multimodal corpora, computional linguistics



  \section{Introduction}

    


  \newpage
  \eightpt
  \bibliographystyle{IEEEtran}

  \bibliography{thebib}

\end{document}
